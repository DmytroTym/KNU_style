\documentclass[ukrainian]{beamer}


\usepackage{cmap}
\usepackage[T2A]{fontenc}
\usepackage[utf8]{inputenc}
\usepackage[main=ukrainian,english,russian]{babel}

\usepackage{accsupp}
\DeclareTextCommand{\cyrii}{T2A}{%
  \BeginAccSupp{method=hex,unicode,ActualText=0456}%
  \symbol{105}%
  \EndAccSupp{}%
}
\DeclareTextCommand{\CYRII}{T2A}{%
  \BeginAccSupp{method=hex,unicode,ActualText=0406}%
  \symbol{73}%
  \EndAccSupp{}%
}

\usetheme{MathDept}

%спеціальні символи
\newcommand {\littleO} {\ensuremath{o}}
\newcommand {\bigO} {\ensuremath{\mathcal{O}}}
\newcommand{\RR}{\mathbb{R}}
\newcommand{\NN}{\mathbb{N}}
\newcommand*\diff{\mathop{}\!\mathrm{d}}
\newcommand{\Dz}{\mathrm{D}_0}
\newcommand{\Iz}{\mathrm{I}_0}
\newcommand{\DzC}{{}^{C}\mathrm{D}_{0}}

\author{Тимоханов Д. О.}
%в квадратних дужках коротка назва, що буде показана знизу сторінок
\title[Рівняння розподіленого порядку]{{\small Випускна кваліфікаційна робота магістра}\\Якісний і чисельний аналіз рівнянь розподіленого порядку}

%тут науковий керівник
\subtitle{Гуляницький А. Л.}


\begin{document}


% Use
%
%     \begin{frame}[allowframebreaks]
%
% if the TOC does not fit one frame.
\begin{frame}
    \frametitle{Зміст}
    \tableofcontents
\end{frame}

\section{Рівняння субдифузії розподіленого порядку}

\begin{frame}
    \frametitle{Рівняння субдифузії розподіленого порядку}
    \tableofcontents[currentsection]
\end{frame}

\begin{frame}
    \frametitle{Рівняння дифузії}
Розглянемо спершу звичне рівняння дифузії:

\begin{equation}\label{diffusion}
\frac{\partial}{\partial{t}}u(x,t)=\sigma^2\cdot\Delta{u}(x,t)+f(x,t)
\end{equation}

Добре відомо, що при $f\equiv0$ середньоквадратичне відхилення частинок є лінійним в часі. На жаль, не всі дифузійні процеси мають цю властивість \cite{Metzler2001}. Тому рівняння (\ref{diffusion}) не підходить для моделювання будь-якої дифузії.

\end{frame}

\begin{frame}
    \frametitle{Рівняння субдифузії}

Зміна фізичних припущень призводить \cite{Metzler2001} до так званого рівняння субдифузії:

\begin{equation}\label{anomalous}
\DzC^{\alpha}u(x,t)=\sigma^2\cdot\Delta{u}(x,t)+f(x,t)
\end{equation}

де $0<\alpha<1$, $\DzC^{\alpha}$ --- оператор дробового диференціювання Капуто:

\begin{equation}\label{caputo}
\DzC^{\alpha}u(x,t)=\frac{1}{\Gamma(1-\alpha)}\int_0^t\frac{u'_s(x,s)}{(t-s)^\alpha}\diff{s},\hspace{2mm}0<\alpha<1
\end{equation}

При $f\equiv0$ рівняння (\ref{anomalous}) призводить до середньоквадратичного відхилення частинок, що зростає в часі як $t^\alpha$. Звідси одразу видно важливу властивість субдифузії --- нелокальність в часі.

\end{frame}

\begin{frame}

\frametitle{Рівняння субдифузії розподіленого порядку}

Але й субдифузійна модель не здатна моделювати всі дифузійні процеси. Одне з узагальнень цієї моделі --- субдифузія розподіленого порядку:

\begin{equation}\label{distr_order}
\int_0^1p(s)\cdot\tau^\alpha\cdot\DzC^{\alpha}u(x,t)\diff\alpha=\sigma^2\cdot\Delta{u}(x,t)+f(x,t)
\end{equation}

тут $p(\cdot)$ --- щільність розподілу порядку похідної. За рахунок вибору цієї щільності можна забезпечити, що середньоквадратичне відхилення частинок зростатиме як $(\log{t})^v$ \cite{}, що узгоджується з експериментальними результатами. Така дифузія називається суперповільною.

\end{frame}

\begin{frame}

\frametitle{Рівняння субдифузії розподіленого порядку}

Розглядатимемо рівняння розподіленого порядку (\ref{distr_order}) разом з початковою та крайовою умовою в області $\Omega\times{I}$:

\begin{equation}\label{initial}
u(x,0)=u_0(x)\hspace{2mm}\forall{x}\in{\Omega}
\end{equation}
\begin{equation}\label{boundary}
u(x,t)|_{x\in\partial\Omega}=0\hspace{2mm}\forall{t}\in{I}
\end{equation}

Про щільність $p(\cdot)$ приймаємо такі припущення:

\begin{itemize}
	
	\vspace{1mm}
	\item $p(\alpha)=p_{reg}(\alpha)+\sum_{i=1}^\infty{p_i}\cdot\delta(\alpha-\alpha_i)$,\smallskip

	\vspace{1mm}
	\item $p_{reg}\in{L^1((0;1))}$, $p_{reg}(\alpha)\geq0$ для майже всіх $\alpha\in(0;1)$, $\forall i\in\NN\hspace{2mm} p_i\geq0$,\smallskip

	\vspace{1mm}
	\item $\exists \epsilon>0:\hspace{1mm}\mathrm{supp}\hspace{1mm}p_{reg}\in[0;1-\epsilon],\hspace{1mm}\forall{i}\in\NN\hspace{2mm}0<\alpha_i<1-\epsilon$,\smallskip

	\vspace{1mm}
	\item $\int_0^1p_{reg}(\alpha)\diff{\alpha}+\sum_{i=1}^\infty{p_i}=1$.\smallskip

	\vspace{1mm}
\end{itemize}

\end{frame}

\section{Простори Соболєва розподіленого порядку}

\begin{frame}
    \frametitle{Простори Соболєва розподіленого порядку}
    \tableofcontents[currentsection]
\end{frame}

\begin{frame}
    \frametitle{Простори Соболєва}

\begin{definition}
Простір Соболєва порядку $\alpha\in(0;1)$:
\begin{equation}
H^\alpha(I)=\big\{v(t)\hspace{1mm}|\hspace{1mm}v\in{L^2(I)};\Dz^\alpha{v}\in{L^2(I)}\big\}
\end{equation}

з нормою
\begin{equation}\label{deriv_sob_norm}
\|v\|_\alpha=\left(\|v\|_{L^2(I)}^2+\|\Dz^\alpha{v}\|_{L^2(I)}^2\right)^\frac{1}{2}
\end{equation}
\end{definition}

в цьому означенні $\Dz^\alpha$ --- оператор диференціювання Рімана-Ліувілля:

\begin{equation}
\Dz^\alpha{u(t)}=\frac{\partial}{\partial{t}}\left(\frac{1}{\Gamma(1-\alpha)}\int_0^t \frac{u(s)}{(t-s)^{\alpha}}\diff{s}\right),\hspace{2mm}0<\alpha<1
\end{equation}

\end{frame}

\begin{frame}
    \frametitle{Простори Соболєва розподіленого порядку}

Природно спробувати узагальнити означення простору Соболєва на розподілені порядки:

\begin{definition}
Простір Соболєва розподіленого порядку зі щільністю $p(\cdot)$:
\begin{equation}
H^{p(\cdot)}(I)=\Big\{v(t)\hspace{1mm}|\hspace{1mm}\forall{\alpha\in\left[0;\frac{\alpha_{max}}{2}\right)}\hspace{1mm}v\in{H^\alpha(I)};\|v\|_{p(\cdot)}<\infty\Big\}
\end{equation}
з нормою
\begin{equation}\label{distr_sob_norm}
\|v\|_{p(\cdot)}^2=\int_0^1p(\alpha)\cdot\|v\|_{\frac{\alpha}{2}}^2\diff{\alpha}
\end{equation}
\end{definition}
\end{frame}

\begin{frame}
\frametitle{Простори розподіленого порядку на шкалі Соболєва}

Позначимо:
\begin{equation}
\alpha_{max}=\max\Big\{\mathrm{esssup}\{\mathrm{supp}\hspace{1mm}p_{reg}(\alpha)\},\sup_{i\in\NN}\alpha_i\Big\}
\end{equation}

З припущень про щільність, $\alpha_{max}<1$.

\begin{proposition}
Мають місце вкладення:

\begin{equation}\label{sob_scale}
\forall{\alpha\in\left[0;\frac{\alpha_{max}}{2}\right)}\hspace{2mm}H^{\frac{\alpha_{max}}{2}}(I)\subseteq{H}^{p(\cdot)}(I)\subseteq H^\alpha(I)
\end{equation}

\end{proposition}

В загальному випадку ці вкладення строгі.
\end{frame}

\begin{frame}
\frametitle{Простори Соболєва розподіленого порядку}

Простори розподіленого порядку мають такі властивості:

\begin{proposition}\label{completeness}
$H^{p(\cdot)}(I)$ є повним простором.
\end{proposition}

\begin{proposition}\label{density}
$C_0^{\infty}(I)$ --- щільна в $H^{p(\cdot)}(I)$ множина.
\end{proposition}

Іншими словами, простір $H^{p(\cdot)}(I)$ можна ототожнити з поповненням $C_0^{\infty}(I)$ за нормою $\|\cdot\|_{p(\cdot)}$.

\end{frame}

\begin{frame}
\frametitle{Альтернативне означення розподіленої норми}

\end{frame}

\section{Виділення}

\begin{frame}
    \frametitle{Виділення}
    
    Можна \alert{виділяти} слова в тексті.
    
    \begin{alertblock}{Важливе повідомлення}
       \alert{Дуже} важливо.
    \end{alertblock}
    
    Можна навіть використати \structure{колір теми}.
\end{frame}

\section{Списки}

\begin{frame}
    \frametitle{Списки}
    
    \begin{itemize}
        \item
        Просто елемент списку.
    \end{itemize}
    
    \begin{enumerate}
        \item
        Нумерований елемент списку.
    \end{enumerate}
    
    \begin{description}
        \item[Важливо]
        підсвічує сірим текстом.
    \end{description}
    
    \begin{example}
        \begin{itemize}
            \item
            Списки змінюють колір після зміни середовища.
        \end{itemize}
    \end{example}
\end{frame}

\section{Список літератури}


\begin{frame}[allowframebreaks]
    \frametitle{Список літератури}

    \begin{thebibliography}{}

        % Article is the default.
        \setbeamertemplate{bibliography item}[book]

        \bibitem{Hartshorne1977}
        R.~Hartshorne.
        \newblock \emph{Algebraic Geometry}.
        \newblock Springer-Verlag, 1977.

        \setbeamertemplate{bibliography item}[article]

        \bibitem{Artin1966}
        M.~Artin.
        \newblock On isolated rational singularities of surfaces.
        \newblock \emph{Amer. J. Math.}, 80(1):129--136, 1966.

        \setbeamertemplate{bibliography item}[online]

        \bibitem{Vakil2006}
        R.~Vakil.
        \newblock \emph{The moduli space of curves and Gromov--Witten theory}, 2006.
        \newblock \url{http://arxiv.org/abs/math/0602347}

        \setbeamertemplate{bibliography item}[triangle]

        \bibitem{AM1969}
        M.~Atiyah og I.~Macdonald.
        \newblock \emph{Introduction to commutative algebra}.
        \newblock Addison-Wesley Publishing Co., Reading, Mass.-London-Don
        Mills, Ont., 1969

        \setbeamertemplate{bibliography item}[text]

        \bibitem{Fraleigh1967}
        J.~Fraleigh.
        \newblock \emph{A first course in abstract algebra}.
        \newblock Addison-Wesley Publishing Co., Reading, Mass.-London-Don Mills, Ont., 1967

    \end{thebibliography}
\end{frame}


\end{document}